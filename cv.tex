% Curriculum Vitae

\documentclass[a4paper, 10pt, oneside]{article}

\usepackage[a4paper, left=0.8in, right=0.8in, bottom=0.5in, top=0.4in]{geometry}

\usepackage[utf8]{inputenc}

% For symbols
\usepackage{amsmath}
\usepackage{amssymb}

\usepackage{multicol}

\usepackage[american]{babel}

\usepackage[autostyle=true]{csquotes}

\usepackage[usenames, dvipsnames]{xcolor}

\usepackage{libertine}

\usepackage{tikz}

\usepackage{hyperref}

% Title (name)
\renewcommand{\title}[1]
{
	\begin{Huge}%
	\textbf{#1}%
	\end{Huge}%
	\par
	\vspace{0.4mm}
}

% Information below the title.
\newenvironment{header}
{
	\setlength{\parskip}{2mm}
	\begin{center}
}
{
	\end{center}
}

% Section box with title.
\renewcommand{\section}[3]
{
  \vspace{-1mm}
	\begin{tikzpicture}
		\draw (#2, #3) node {\uppercase{\textbf{#1}}};
		\draw [thick] (0, 0) rectangle (6.65in, 0.8);
	\end{tikzpicture}
}

% Starts section entry.
\newenvironment{entry}[3]
{
	\vspace{-5mm}
	\begin{itemize}
		\item #1\hfill#3
		\vspace{1mm}
		#2
		\begin{itemize}
}
{
		\end{itemize}
	\end{itemize}
  \vspace{-1mm}
}

% Fancyheader will complain else.
\setlength{\headheight}{15pt}

% No indent at the start of a new paragraph.
\setlength{\parindent}{0pt}

% Adds an extra newline to paragraphs.
\addtolength{\parskip}{\baselineskip}

% Separator for the title.
\newcommand{\separate}{\, {\scriptsize $\bullet$} \,}

% Underlines urls.
\newcommand{\link}[2]{\underline{\href{#2}{#1}}}

% Section entry list-item: first level.
\renewcommand{\labelitemi}{$\bullet$}

% Section entry list-item: second level.
\renewcommand{\labelitemii}{\large$\circ$}


\begin{document}
% No page numbers
\pagenumbering{gobble}

\begin{header}
	\title{Peter Goldsborough}
	\link{peter@goldsborough.me}{mailto:peter@goldsborough.me}
	\separate
	\link{linkedin.com/in/petergoldsborough}{http://www.linkedin.com/in/petergoldsborough}
	\separate
	\link{goldsborough.me}{http://goldsborough.me}
\end{header}

\section{Work Experience}{2}{0.38}

% \begin{entry}
% 	{\link{\textbf{ETH Z\"{u}rich}}{http://twitter.com/}, Zurich, Switzerland}
% 	{\\Research Intern, Systems Group}
% 	{05/2017 --- Present}
%   \item Deep Learning research for astrophysical simulations.
% 	\item Worked on Generative Adversarial Networks for image superresolution.
% \end{entry}

% \begin{entry}
% 	{\link{\textbf{University of Illinois at Urbana-Champaign}}{http://twitter.com/}, Illinois, USA}
% 	{\\Research Intern, Compiler Group}
% 	{05/2017 --- Present}
% 	\item Applied machine learning to compiler optimization in the LLVM research group.
%   \item Contributed to LLVM and clang projects and learnt about compilers.
% \end{entry}

% \begin{entry}
% 	{\link{\textbf{University of Oxford}}{http://twitter.com/}, Oxford, UK}
% 	{\\Research Intern, Robotics Institute}
% 	{08/2017 --- Present}
% 	\item Worked on deep reinforcement learning algorithms for autonomous motion planning.
%   \item Implemented image segmentation for robotics.
%   \item Learnt about reinforcement learning and robotics.
% \end{entry}

% \begin{entry}
% 	{\link{\textbf{Broad Institute of MIT and Harvard}}{http://twitter.com/}, Greater Boston Area, USA}
% 	{\\Research Intern, Imaging Group}
% 	{05/2017 --- Present}
% 	\item Deep learning research for cell analysis and medical imaging.
%   \item Worked on CellProfiler, a cell analysis platform.
%   \item Distributed workloads across dozens of nodes.
% \end{entry}

\begin{entry}
	{\link{\textbf{Facebook}}{http://twitter.com/}, London, UK}
	{\\Intern, Real Time Systems}
	{05/2017 --- Present}
  \vspace{-1mm}
	\item Optimizing highly distributed real time infrastructure at the core of Facebook.
\end{entry}

\begin{entry}
	{\link{\textbf{Mindi}}{http://mindi.io/}, London, UK}
	{\\Research Intern}
	{04/2017 --- 05/2017}
  \vspace{-1mm}
  \item Deep reinforcement learning for data center scheduling and load balancing.
  \item Researched time series prediction with (recurrent) neural networks.
\end{entry}

\begin{entry}
	{\link{\textbf{Bloomberg}}{http://bloomberg.com/}, London, UK}
	{\\Intern, Instant Bloomberg}
	{11/2016 --- 04/2017}
  \vspace{-1mm}
  \item Worked on distributed message tracing in the Instant Bloomberg (IB) messaging system.
  \item Wrote a network traffic simulation tool that produces messages to Apache Kafka message queue clusters.
\end{entry}

\begin{entry}
  {\link{\textbf{Google}}{http://duckduckgo.com/}, London, UK}
	{\\Intern, gTech}
	{08/2016 --- 11/2016}
  \vspace{-1mm}
  \item Built chatbots in Go, using natural language processing.
  \item Created a web platform to showcase Google's ad technologies.
\end{entry}

\begin{entry}
  {\link{\textbf{Technical University Munich}}{https://www-db.in.tum.de}, Germany}
	{\\Research Assistant, Chair for Database Systems}
	{04/2016 --- 09/2016}
  \vspace{-1mm}
  \item Investigated interprocess communication techniques for low-latency transmission of database queries.
  \item Wrote a software library that replaces domain sockets by injecting a shared memory transmission channel.
\end{entry}

\begin{entry}
  {\link{\textbf{Klagenfurt University}}{https://nes.aau.at}, Austria}
	{\\Research Intern, Institute of Networked and Embedded Systems}
	{10/2014 --- 07/2016}
  \vspace{-1mm}
  \item Applied machine learning to Non-Intrusive-Load-Monitoring (NILM) in Python and C++.
	\item Invented custom $O(N \log N)$ clustering algorithm to replace existing $O(N^2)$ solution.
\end{entry}
\vspace{-1mm}

\section{Projects}{1.1}{0.37}
\vspace{-4mm}
\begin{itemize}
  \item Lead a team of 12 students to develop \link{an architecture-independent assembly simulator}{https://github.com/TUM-LRR/era-gp-sim} in C++14 and Qt5 supporting RISC-V, x86 and ARM ISAs.
  \item \link{clang-expand}{https://github.com/goldsborough/clang-expand} is a clang and LLVM based tool to inline function calls and expand macros in C, C++ and Objective-C for visual benefit and easier refactoring. Featured in \link{LLVM Weekly 169}{http://llvmweekly.org/issue/169}.
  \item \link{lru-cache}{https://github.com/goldsborough/lru-cache} is a least-recently-used (LRU) cache implementation in C++.
  \item Talks on \emph{Deep Learning with TensorFlow} at \link{PyCon UK}{http://2016.pyconuk.org/talks/an-introduction-to-deep-learning-with-tensorflow/}, \link{Python Munich}{https://www.meetup.com/Munchen-Python-Data/events/231889942/} and PyData London.
  \item All my projects can be found at
	\link{github.com/goldsborough}{https://www.github.com/goldsborough}.
\end{itemize}

\section{Education}{1.3}{0.38}
\vspace{0.1cm}
\begin{entry}
	{\link{\textbf{Technical University of Munich (TUM)}}{http://www.in.tum.de/en}, Germany}
	{\\B.Sc. in Computer Science}
	{10/2015 --- Present}
  \item Top 5\% in all courses.
  \item Awarded German National Scholarship (1\% of applicants admitted).
\end{entry}

\section{Publications}{1.5}{0.38}
\vspace{-2mm}
\begin{itemize}
  \item \emph{A Tour of TensorFlow}, Peter Goldsborough, Aug. 2016 --- \link{arXiv:1610.01178}{https://arxiv.org/abs/1610.01178}
  \item \emph{NILM: A Review and Outlook}, Christoph Klemenjak, Peter Goldsborough, Sep. 2016 --- \link{arXiv:1610.01191}{https://arxiv.org/abs/1610.01191}
\end{itemize}

\end{document}
