% Curriculum Vitae

\documentclass[a4paper, 10pt, oneside]{article}

\usepackage[a4paper, left=0.8in, right=0.8in, bottom=0.5in, top=0.4in]{geometry}

\usepackage[utf8]{inputenc}

% For symbols
\usepackage{amsmath}
\usepackage{amssymb}

\usepackage{multicol}

\usepackage[american]{babel}

\usepackage[autostyle=true]{csquotes}

\usepackage[usenames, dvipsnames]{xcolor}

\usepackage{libertine}

\usepackage{tikz}

\usepackage{hyperref}

% Title (name)
\renewcommand{\title}[1]
{
	\begin{Huge}%
	\textbf{#1}%
	\end{Huge}%
	\par
	\vspace{0.4mm}
}

% Information below the title.
\newenvironment{header}
{
	\setlength{\parskip}{2mm}
	\begin{center}
}
{
	\end{center}
}

% Section box with title.
\renewcommand{\section}[3]
{
  \vspace{-1mm}
	\begin{tikzpicture}
		\draw (#2, #3) node {\uppercase{\textbf{#1}}};
		\draw [thick] (0, 0) rectangle (6.65in, 0.8);
	\end{tikzpicture}
}

% Starts section entry.
\newenvironment{entry}[3]
{
	\vspace{-5mm}
	\begin{itemize}
		\item #1\hfill#3
		\vspace{1mm}
		#2
		\begin{itemize}
}
{
		\end{itemize}
	\end{itemize}
  \vspace{-1mm}
}

% Fancyheader will complain else.
\setlength{\headheight}{15pt}

% No indent at the start of a new paragraph.
\setlength{\parindent}{0pt}

% Adds an extra newline to paragraphs.
\addtolength{\parskip}{\baselineskip}

% Separator for the title.
\newcommand{\separate}{\, {\scriptsize $\bullet$} \,}

% Underlines urls.
\newcommand{\link}[2]{\underline{\href{#2}{#1}}}

% Section entry list-item: first level.
\renewcommand{\labelitemi}{$\bullet$}

% Section entry list-item: second level.
\renewcommand{\labelitemii}{\large$\circ$}


\begin{document}
% No page numbers
\pagenumbering{gobble}

\begin{header}
	\title{Peter Goldsborough}
	\link{peter@goldsborough.me}{mailto:peter@goldsborough.me}
	\separate
	\link{linkedin.com/in/petergoldsborough}{http://www.linkedin.com/in/petergoldsborough}
	\separate
	\link{www.goldsborough.me}{http://goldsborough.me}
\end{header}

\section{Work Experience}{2}{0.38}

\begin{entry}
	{\link{\textbf{Facebook}}{http://twitter.com/}, London, United Kingdom}
	{\\Intern}
	{05/2017 --- 08/2017}
	\item Team and project yet to be decided.
\end{entry}

\begin{entry}
	{\link{\textbf{Bloomberg}}{http://bloomberg.com/}, London, United Kingdom}
	{\\Intern}
	{11/2016 --- 04/2017}
	\item Working on message tracing in highly distributed chat systems.
\end{entry}

\begin{entry}
	{\link{\textbf{Google}}{http://duckduckgo.com/}, London, United Kingdom}
	{\\Intern, gTech Team}
	{08/2016 --- 11/2016}
  \item Developed a web platform to showcase Google’s ad technologies.
  \item Worked on a chatbot system that makes remote procedure calls to an internal issue tracker to allow natural language bug queries.
  \item Open-sourced an integration of the Google Publisher Tags (GPT) library with Angular under an official Google GitHub organization.
  \item Learnt JavaScript, AngularJS, Go, RPC and Protobuf from scratch.
\end{entry}

\begin{entry}
	{\link{\textbf{Technische Universität München}}{https://www-db.in.tum.de}, Germany}
	{\\Research Assistant, Chair for Database Systems}
	{04/2016 --- 09/2016}

  \item Investigated the efficiency of various interprocess communication techniques for low-latency transmission of database queries.
  \item Implemented a software library that transparently replaces domain sockets by injecting a custom shared memory transmission channel.
  \item This can speed up a variety of applications by an order of magnitude.
  \item Wrote more than 10,000 lines of low level C code.
\end{entry}

\begin{entry}
	{\link{\textbf{Institute of Networked and Embedded Systems}}{https://nes.aau.at}, Klagenfurt University, Austria}
	{\\Research Intern}
	{10/2014 --- 07/2016}

\item Applied machine-learning and data-science techniques to make NILM
  algorithms unsupervised.
	\item Invented custom $O(N \log N)$ clustering algorithm to replace existing $O(N^2)$ solution.
	\item Wrote 8363 lines of C++ code (working 5-10 hours/week)
\end{entry}

\begin{entry}
	{\link{\textbf{Institute of Networked and Embedded Systems}}{https://nes.aau.at}, Klagenfurt University, Austria}
	{\\Summer Intern}
	{07/2014}

	\item Ported Non-Intrusive-Appliance-Monitoring (NIALM) algorithms from MATLAB to Python, C++ and SQL.
	\item Visualized real-time appliance energy data in a Heroku cloud app using Python, Flask, PostgreSQL, HTML, CSS and Javascript; still live at \link{nilm.herokuapp.com}{http://nilm.herokuapp.com}.
\end{entry}

\vspace{-2mm}
\section{Education}{1.3}{0.38}
\begin{entry}
	{\link{\textbf{BG $|$ BRG St. Martin High School}}{http://www.it-gymnasium.at/index.php?id=715t}, Villach, Austria}{}
	{09/2011 --- 06/2015}
	\item Graduated as valedictorian with straight A's across all subjects and my thesis.
  \item 120-page diploma thesis on \link{\emph{Developing a Digital Synthesizer in C++}} \,, investigating the mathematics, digital signal processing and software engineering behind digital music synthesizers.
	\item School president $2013/2014$; long-time involvement in student representation.
  \item 3\textsuperscript{rd} place in state rhetoric competition.
  \item 4\textsuperscript{th} place in state philosophy contest.
  \item Participated in the \link{Atlas Shrugged essay competition}{http://essaycontest.aynrandnovels.com/AtlasShrugged.aspx} organized by the Ayn Rand Institute, writing on the topic of laissez-faire capitalism and objectivist philosophy.
\end{entry}

\pagebreak
\begin{entry}
	{\link{\textbf{Technical University of Munich (TUM)}}{http://www.in.tum.de/en}, Germany}
	{\\B.Sc. in Computer Science}
	{10/2015 --- Present}
  \item Highlights:
  \begin{itemize}
    \item Admitted with a score of $100/100$.
    \item Top 5\% in every course.
    \item Leading a team of 12 fellow student developers in the \emph{Computer Architecture} practicum (\emph{Großpraktikum}). We are developing an architecture-independent assembly simulator. At most 20 out of almost 1000 students are selected for this practicum. Next to regular development tasks I am responsible for the (agile) coordination of the team's workflow and time schedule. Also, I support all teams with technical expertise in C++ and general software development.
    \item Awarded German National Scholarship (Deutschlandstipendium).
  \end{itemize}
  \item First Semester:
  \begin{itemize}
    \item Discrete Mathematics; 95th percentile.
    \item Introduction to Database Systems; 99th percentile.
    \item Introduction to Computer Architecture; 97th percentile.
    \item Introduction to Computer Science; 98th percentile.
    \item Fundamentals of Programming: 99th percentile (291.5/292 points).
  \end{itemize}
  \item Second Semester:
  \begin{itemize}
    \item Linear Algebra; 99th percentile.
    \item Introduction to Computer Networks and Distributed Systems: 99th percentile.
    \item Introduction to Algorithms and Data Structures; 90th percentile, 96\% on assignments.
    \item Machine Learning Seminar; Topic: ``Deep Learning with TensorFlow''; Grade: A.
    \item Computer Architecture Practicum (Top 20 of ``Intro. to Computer Architecture'' are selected); spans 2 semesters.
  \end{itemize}
\end{entry}

\vspace{-2mm}
\begin{entry}{\textbf{Online Education}}{}{07/2013 --- Present}
  \item I have taken a number of online courses on Coursera, Udacity, MIT OpenCourseWare next to my regular education to further my horizon and dive deeper into topics that interest me.
  \item Coursera:
  \begin{itemize}
    \item Algorithms I (Princeton University)
    \item Algorithms II (Princeton University)
    \item Machine Learning (Stanford University)
    \item Learning How To Learn (University of California, San Diego)
  \end{itemize}
  \item Udacity:
  \begin{itemize}
    \item Machine Learning nano-degree (organized by Google)
    \item Deep Learning (organized by Google)
  \end{itemize}
  \item MIT OpenCourseWare:
  \begin{itemize}
    \item Various lectures on efficient algorithms and data structures on undergraduate and graduate level.
  \end{itemize}
\end{entry}

\section{Skills}{0.9}{0.4}
\vspace{-2mm}
\begin{multicols}{2}
	\begin{itemize}
    \item Very Good: C++, Python, Java, C
		\item Good: \LaTeX\footnote{See CV. \emph{Beware}: recursion.}, HTML,
      CSS/SASS, Machine Learning
		\item Intermediate: Git, Assembly, JavaScript, SQL, Qt
		\item Prior Experience: VHDL, Arduino, AVR C
		\item (Operating) Systems: \textbf{OS X, Linux}, Windows, Arduino, ATMEL $\mu$-controllers
    \\
	\end{itemize}
\end{multicols}

\section{Projects}{1.1}{0.37}
\vspace{-5mm}
\begin{itemize}
	\item Various projects found at
	\link{github.com/goldsborough}{https://www.github.com/goldsborough}; snippets and scripts at \link{gist.github.com/goldsborough}{https://gist.github.com/goldsborough}
	\item Personal blog \& website, old at \link{thecodeinn.blogspot.com}{http://www.thecodeinn.blogspot.com}, new at \link{goldsborough.me}{http://www.goldsborough.me}.
  \item From February 5\textsuperscript{th} to 6\textsuperscript{th} 2015 I head organized a two-day session of the European Youth Parliament (EYP) in Villach, Austria. I managed a team of five organizers, together with whom I raised over \EURdig{}1300 in public and private funding, negotiated deals with McDonalds and Subway to supply us with food and beverages for our coffee and lunch breaks -- in exchange for various forms of advertisement -- and raised enough interest to get our session on \link{national television}{https://youtu.be/YuwF6MseivQ?t=3m42s} and into various regional newspapers.
	% \item I worked on a four-part freelance tutorial project \emph{Building a Text Editor with
  %     PyQt} \link{published on Binpress}{https://goo.gl/XLZl4z}, currently with
  %   almost 70 stars \link{on GitHub}{https://goo.gl/wbRcIY}. It was mentioned in PythonWeekly (issue 155) and PyCoder's Weekly (issues 131 and 132) newsletters.
  \item I'm leading a team to develop \link{an architecture-independent assembly simulator}{https://github.com/TUM-LRR/era-gp-sim} in C++14 and Qt5 for educational and scientific purposes. We are primarily focusing on RISC-V as an open-source Instruction Set Architecture (ISA), while keeping all technical doors open to support other architectures such as x86 or ARMv7. Next to development, I am responsible for coordination and management of the team.
  \item \link{Mavrchester}{https://github.com/goldsborough/Mavrchester} is a C library implementing the Manchester Encoding protocol for embedded AVR microcontrollers. Manchester Encoding is a physical-layer communication protocol used for Ethernet and wireless radio frequency transmission. I used this library to build a small wireless weather station on an Atmel microcontroller.
  \item \link{latexpp}{https://github.com/goldsborough/latexpp} is a first-of-its kind C++ library for generating LaTeX equations via C++. It supports conversion of LaTeX equations like \texttt{\textbackslash frac\{1\}\{2\}} to HTML as well as JPG, PNG and SVG images. I made it using Google's V8 JavaScript engine to write JavaScript from C++.
  \item \link{dispy}{https://github.com/goldsborough/dispy} is a tiny Python bytecode disassembler that can take any piece of Python code (function, method, class etc.) and prints out the underlying cPython bytecode instructions and arguments making up each line of code.
  \item \link{lru-cache}{https://github.com/goldsborough/lru-cache} is a feature-complete least-recently-used (LRU) cache implementation in modern C++ that allows for efficient function memoization while avoiding a memory blowup. I additionally implemented a timed LRU cache with a \emph{time to live} for elements, ideal for server-side resource caching
\end{itemize}

\section{Organizations}{1.7}{0.4}
\begin{entry}{\link{European Youth Parliament}{http://eyp.at}}{}{07/2013 --- Present}
  \item The European Youth Parliament (EYP), is politically-motivated, non-governmental organization attempting to spread political awareness among youths.
  \item There are regular meetings all over Europe concerning current EU-related political topics, such as environmental sustainability, regional development or fiscal policy.
  \item These topics are discussed and debated in a simulation of European Parliament.
  \item I have participated in more than ten such sessions as delegate, chairperson and head organizer.
  \item For two years I held the position of Regional Coordinator for the Austrian state of Carinthia.
\end{entry}
\begin{entry}{\link{Model United Nations}{http://euromun.org}}{}{07/2013 --- Present}
  \item Model United Nations is an international association dedicated to simulating the United Nations and allowing youths to directly participate in topics of diplomacy and international relations.
  \item In May of 2016, I represented Spain as part of the NATO committee in Maastricht, The Netherlands, discussing subjects such as cyber-warfare and the militarization of the Arctic.
\end{entry}
\begin{entry}{\link{TU Investment Club}{https://tuinvest.de}}{\\Algorithmic Trading Team}{10/2015 --- Present}
  \item The TU Investment Club is a non-profit student organization at Technical University of Munich dedicated to the education of students with a distinct interest in financial markets.
  \item I am part of the algorithmic trading team, bringing together computer science and financial knowledge to explore what impact we can have on the financial market.
  \item I have been a member of the interviewing committee for applicants with a technical background.
\end{entry}
\begin{entry}{\link{ACM Student Chapter}{https://acmmunich.de}}{}{07/2016 --- Present}
  \item The Association for Computing Machinery (ACM) Student Chapter at TUM is a group of bachelor's and master's students passionate about computer science, scientific computing and machine learning.
  \item We organize regular events with more than 200 attendees with rockstar speakers from the field of artificial intelligence and general computer science.
  \item I have managed the invitation and organization of our speakers twice, once Prof. Dr. Sepp Hochreiter and once an engineer from Google DeepMind.
\end{entry}

\section{Speaking}{1.25}{0.4}
\begin{simpleentry}
  \item Introduction to Machine Learning with TensorFlow @ \link{Python Meetup Munich}{http://www.meetup.com/PyMunich/events/232415909/}. \hspace{3cm} 07/2016
  \item Deep Learning with TensorFlow @ \link{PyCon UK}{https://www.youtube.com/watch?v=2vS9vHlPGoE}. \hspace{7.65cm} 09/2016
  \item Deep Learning with TensorFlow @ \link{Cambridge Coding Academy}{https://www.eventbrite.co.uk/e/data-science-tech-talk-an-introduction-to-deep-learning-with-tensorflow-tickets-29013480100}. \hspace{5.03cm} 11/2016
  \item Deep Learning with TensorFlow @ \link{PyData London}{https://www.meetup.com/PyData-London-Meetup/events/235429712/}. \hspace{6.95cm} 12/2016
\end{simpleentry}

\section{Publications}{1.65}{0.4}
\begin{simpleentry}
  \item Christoph Klemenjak, Peter Goldsborough, Non Intrusive Load Monitoring: A Review and Outlook, Sep. 2016. Preprint: \link{arXiv:1610.01191}{https://arxiv.org/abs/1610.01191}
  \item \emph{A Tour of TensorFlow} (2016), published as part of the Data Mining Seminar at TUM. Available: \link{arxiv:1610.01178}{https://arxiv.org/abs/1610.01178}
\end{simpleentry}

\section{Languages}{1.6}{0.4}
\begin{simpleentry}
  \item German: Native
  \item English: Native
  \item Nerd: Native
  \item French: Limited working proficiency
\end{simpleentry}

\section{Certifications}{1.8}{0.4}
\begin{simpleentry}
  \item Cisco CCNA Discovery: Networking for Home and Small Businesses
  \item Cisco CCNA Discovery: Working at a Small-to-Medium Business or ISP
  \item Certificate in Advanced English (CAE)
  \item Diplôme d'études en langue française (DELF)
\end{simpleentry}

\section{Interests}{1.3}{0.4}
\begin{simpleentry}
  \item International politics and diplomacy
  \item Philosophy and differing schools of thought
  \item Entrepreneurship and startup culture
  \item 20th century British and US literature
  \item Self development and learning something new every day
  \item Leadership and inspiring people to maximize their potential
  \item Artificial intelligence and machine, especially deep, learning
  \item Beautiful, expressive, well formatted and human readable code
\end{simpleentry}

% \vspace{0.1cm}
% \begin{quote}
%   \textbf{\textsc{Congratulations}} and especially thank you, for making it all the way to the end of my CV. I hope you enjoyed the journey. I'm confident this document will grow and mature with time. I consider myself on a mission to leave a large and meaningful impact on society. This is and will be reflected here, in one or many ways, soon.
% \end{quote}
%
% $$\varheart$$

\end{document}
