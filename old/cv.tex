% !TEX TS-program = xelatex
% !TEX encoding = UTF-8

\documentclass[a4paper, 10pt, oneside]{article}

\usepackage[a4paper, left=0.8in, right=0.8in, bottom=0.5in, top=0.4in]{geometry}

\usepackage[utf8]{inputenc}

% For symbols
\usepackage{amsmath}
\usepackage{amssymb}

\usepackage{multicol}

\usepackage[american]{babel}

\usepackage[autostyle=true]{csquotes}

\usepackage[usenames, dvipsnames]{xcolor}

\usepackage{libertine}

\usepackage{tikz}

\usepackage{hyperref}

% Title (name)
\renewcommand{\title}[1]
{
	\begin{Huge}%
	\textbf{#1}%
	\end{Huge}%
	\par
	\vspace{0.4mm}
}

% Information below the title.
\newenvironment{header}
{
	\setlength{\parskip}{2mm}
	\begin{center}
}
{
	\end{center}
}

% Section box with title.
\renewcommand{\section}[3]
{
  \vspace{-1mm}
	\begin{tikzpicture}
		\draw (#2, #3) node {\uppercase{\textbf{#1}}};
		\draw [thick] (0, 0) rectangle (6.65in, 0.8);
	\end{tikzpicture}
}

% Starts section entry.
\newenvironment{entry}[3]
{
	\vspace{-5mm}
	\begin{itemize}
		\item #1\hfill#3
		\vspace{1mm}
		#2
		\begin{itemize}
}
{
		\end{itemize}
	\end{itemize}
  \vspace{-1mm}
}

% Fancyheader will complain else.
\setlength{\headheight}{15pt}

% No indent at the start of a new paragraph.
\setlength{\parindent}{0pt}

% Adds an extra newline to paragraphs.
\addtolength{\parskip}{\baselineskip}

% Separator for the title.
\newcommand{\separate}{\, {\scriptsize $\bullet$} \,}

% Underlines urls.
\newcommand{\link}[2]{\underline{\href{#2}{#1}}}

% Section entry list-item: first level.
\renewcommand{\labelitemi}{$\bullet$}

% Section entry list-item: second level.
\renewcommand{\labelitemii}{\large$\circ$}


\begin{document}

%%%%%%%%%%%%%%%
% TITLE BLOCK %
%%%%%%%%%%%%%%%

\title{Peter Goldsborough}

\begin{subtitle}
\href{https://www.google.com/maps/place/46°37'42.6%22N+13°49'28.0%22E}
{Untere Fellacher Str. 46, 9500 Villach, Austria}
\, \BulletSymbol\,
+43\,(0)\,676\,785\,7700
\par
\href{mailto:petergoldsborough@hotmail.com}
{petergoldsborough@hotmail.com}
\,\BulletSymbol\,
\href{http://www.thecodeinn.blogspot.com}
{http://www.thecodeinn.blogspot.com}
\end{subtitle}

\begin{body}

%%%%%%%%%%%%%%%
%% EDUCATION %%
%%%%%%%%%%%%%%%

\section{Education}
{Education}

\href{http://it-gymnasium.at}
{\textbf{BG | BRG St.Martin}},
Villach, Austria
\hfill
\Datestamp{2007}{09}{01} --
\Datestamp{2015}{07}{01} (expected)
\begin{detail}

Thesis:
\href{http://issuu.com/petergoldsborough/docs/thesis}
{Developing a Digital Synthesizer in C++}
\par
Supervisor: Professor Martin Kastner
\par
Grade Average (2013/2014):
1.1 \footnote{The Austrian school system grades from 1 (best) to 5 (worst); 15 subjects. Top of class.}
\par
Special Mention: School President (2013/2014)

\begin{multicols}{2}
Additional Classes:
\begin{itemize}
  \item Advanced Mathematics
  \item Applied Physics
  \item Cisco Networking
\end{itemize}

Organizations and Activities:
\begin{itemize}
  \item Student Representatives Team
  \item Rhetoric, Debate and Philosophy Team
  \item Information Technology Team
\end{itemize}
\end{multicols}
\end{detail}

%%%%%%%%%%%%%%%%%%%%%
%% WORK EXPERIENCE %%
%%%%%%%%%%%%%%%%%%%%%

\section{Work Experience}
{Work Experience}

\href{https://nes.aau.at}
{\textbf{Institute of Networked and Embedded Systems}},
University of Klagenfurt, Austria

Research Intern
\hfill \Datestamp{2014}{10}{01} -- Present

\begin{detail}
After a successful summer internship I was employed as a part-time research intern at the Institute for Networked and Embedded Systems at the University of Klagenfurt. My responsibilities include the porting of an existing MATLAB Non-Intrusive-Appliance-Load-Monitoring (NIALM) algorithm and associated Python codebase to C++ for performance gains as well as the extraction, processing and visualization of appliance power data using the UDOO development board and the PALDI load-disaggregation algorithm. The technologies I use include C++, Python, MATLAB, SQLite3, Bash, Gnuplot, JSON and \LaTeX{}.
\end{detail}

Summer Intern
\hfill \Datestamp{2014}{07}{01}

\begin{detail}
I had my first work experience at the Institute for Networked and Embedded Systems at the University of Klagenfurt in July 2014, where I worked on Non-Intrusive-Appliance-Load-Monitoring (NIALM) algorithms based on real-time Particle Filtering on a UDOO development Board. I implemented the load-disaggregation algorithms with Python, C++ and SQLite3; visualized them in a web-app with Python, Flask, Heroku, PostgreSQL, HTML, CSS and Javascript and sourced appliance data using Arduino, C and Bash Shell Script. My final report was written with the \LaTeX{} document preparation system.
\end{detail}

%%%%%%%%%%%%%%%%%%%
%% ORGANIZATIONS %%
%%%%%%%%%%%%%%%%%%%

\section{Organizations}
{Organizations}

\textbf{European Youth Parliament}

Regional Coordinator
\hfill \Datestamp{2013}{07}{01} -- Present

\begin{detail}
The European Youth Parliament (EYP) is a politically-motivated, non-governmental organization attempting to spread political awareness among young people. There are regular meetings of varying priority and duration all over Europe concerning current EU-related topics. These topics are discussed and debated in a parliament simulation. As I see it, the European Youth Parliament is identical to the ``real'' European Parliament --- we are just more productive. Since July 2013 I hold the position of Regional Coordinator for my state, Carinthia. My responsibilities include, among others, the coordination and organization of all regional sessions and meetings, as well as being the go-to person for all requests or inquiries regarding the EYP in Carinthia.
\end{detail}

\textbf{Student Representatives Team, BG | BRG St.Martin}

School President \hfill \Datestamp{2013}{09}{10} -- \Datestamp{2014}{07}{10}

\begin{detail}
I was elected school president for the 2013/2014 academic year. For my period one of my primary aims was to bring the team of student representatives closer to the students. I put in place monthly meetings between representatives of our school's 48 classes, my two deputies and me, to discuss previous, current and future actions of the team. Another big concern of mine was to update our ridiculously antiquated IT infrastructure. After negotiations with our head mistress, I managed to get our school administration to buy 32 new computer installations worth 60 000 Euros.
\end{detail}

%%%%%%%%%%%%%%
%% PROJECTS %%
%%%%%%%%%%%%%%

\section{Projects}
{Projects}

\textbf{Blog}

\begin{detail}
Since July 2013 I have my own blog, on which I regularly post tutorials and articles on a variety of topics, as well as projects or code that I explain and make available for other people to learn from and use. I see it as a digital archive and showcase of my projects and development as well as my active participation in the open source movement.

Link: \href{http://thecodeinn.blogspot.com}{http://thecodeinn.blogspot.com}
\end{detail}

\textbf{EYP Day Villach 2015}

\begin{detail}
From September 2014 to February 2015 I was the Head-Organizer of the European Youth Parliament (EYP) Day in Villach, Austria, which took place on the 5th and 6th of February 2015. I managed a team of five organizers, together with whom I raised over 1300 Euros in public and private funding --- e.g. from Infineon Technologies or the Carinthian International Club; negotiated deals with McDonalds and Subway to supply us with food and beverages for our coffee and lunch breaks -- in exchange for various forms of advertisement -- and raised enough interest to get our session on national television.
Session page: \href{http://www.eypaustria.org/eypdays/eyp-day-villach/}{http://www.eypaustria.org/eypdays/eyp-day-villach/}
\par
TV link (relevant part at around 1m:35s; URL shortened for length): \href{http://goo.gl/XBzv0g}{http://goo.gl/XBzv0g}
\end{detail}

\textbf{Building a Text Editor with PyQt}

\begin{detail}
In August and September of 2014 I worked on a 4-part freelance tutorial project, titled ``Building a text editor with PyQt'', for which I built a text editor with the Python GUI framework PyQt. All four parts were published on Binpress, the marketplace for commercial and open-source software. According to Alexis Santos, the editor of Binpress, the series has become one of the most popular in its category and was shared more than a hundred times on Twitter, Facebook, Reddit and LinkedIn. Moreover, it was featured in two of the most widespread Python Newsletters --- PythonWeekly (issue 155) and PyCoder's Weekly (issues 131 and 132) --- and has been starred 25+ times on GitHub.

Binpress link: \href{http://binpress.com/tutorial/building-a-text-editor-with-pyqt-part-one/143}{http://binpress.com/tutorial/building-a-text-editor-with-pyqt-part-one/143}
\par
Github tutorial repository: \href{https://github.com/goldsborough/Writer-Tutorial}{https://github.com/goldsborough/Writer-Tutorial}
\end{detail}

\textbf{Developing a Digital Synthesizer in C++}

\begin{detail}
Between January 2014 and February 2015 I worked on my final academic year thesis, ``Developing a Digital Synthesizer in C++'', in which I describe the theoretical concepts behind and implementation of a digital music synthesizer. The knowledge of mathematics, digital signal processing, audio programming and computer science which I acquired through this project, paired together with around 10 000 lines of C++, Qt, Python and other code, paved the way for \emph{Anthem} --- an open-source, C++ Frequency Modulation Synthesizer.

Thesis (written with \LaTeX): \href{http://issuu.com/petergoldsborough/docs/thesis}{http://issuu.com/petergoldsborough/docs/thesis}
\par

\emph{Anthem} GitHub project page: \href{https://github.com/goldsborough/Anthem}{https://github.com/goldsborough/Anthem}
\end{detail}



\textbf{GitHub}

\begin{detail}
Most of my current and past projects using a variety of different technologies can be found on GitHub, either as repositories, for larger projects, or as gists, for smaller programs, scripts or snippets.

Repositories: \href{https://github.com/goldsborough}{https://github.com/goldsborough}
\par
Gists: \href{https://gist.github.com/goldsborough}{https://gist.github.com/goldsborough}
\end{detail}


%% HARD SKILLS %%
%%%%%%%%%%%%%%%%%

\section{Hard Skills}
{Hard Skills}

C/C++, Qt, Python, PyQt, \LaTeX
\par
HTML, CSS, SASS, JavaScript, SQL, Flask, Heroku
\par
UNIX, Linux, Windows, Git, Digital Signal Processing
\par
Arduino, Atmel AVR, Electronics, Raspberry Pi, UDOO
\par
<Insert anything I need to pick up for my current project>

%%%%%%%%%%%%%%%%%
%% SOFT SKILLS %%
%%%%%%%%%%%%%%%%%

\section{Soft Skills}
{Soft Skills}

Team player by day, lone wolf by night
\par
Public Speaking, Rhetoric, Debate
\par
Leadership, Organization, Decisiveness
\par
Ambition, Perfectionism, Hard-working

%%%%%%%%%%%%%%%%%%%%
%% CERTIFICATIONS %%
%%%%%%%%%%%%%%%%%%%%

\section{Certifications}
{Certifications}

Cisco CCNA Discovery: Networking for Home and Small Businesses
\par
Cisco CCNA Discovery: Working at a Small-to-Medium Business or ISP
\par
Certificate in Advanced English (CAE)
\par
Dipl\^{o}me d'\'{E}tudes en Langue Fran\c{c}aise (DELF)

%%%%%%%%%%%%%%%
%% INTERESTS %%
%%%%%%%%%%%%%%%

\section{Interests}
{Interests}

Computer Science, Electronic Music, Entrepreneurship, Blogging (\href{http://thecodeinn.blogspot.com}{http://thecodeinn.blogspot.com}), Open Source Movement, Ayn Rand and Objectivism, 20\textsuperscript{th} century literature, Skiing, Participation in and organization of European Youth Parliament Sessions, Playing icehockey

%%%%%%%%%%%%%%%
%% LANGUAGES %%
%%%%%%%%%%%%%%%

\section{Languages}
{Languages}

English: Native language
\par
German: Native language
\par
French: Intermediate

%%%%%%%%%%%%%%
%% PERSONAL %%
%%%%%%%%%%%%%%

\section{Personal Information}
{Personal Information}

Born: July 26, 1997 in Villach, Austria
\par
Citizenships: Austria, France, USA

\end{body}

\end{document}
