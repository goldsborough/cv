\documentclass[a4paper,10pt,oneside]{article}

% Do not stop on errors during compilation.
\nonstopmode

%%%%%%%%%%%%%%%%%%%%%%%%%%%%%%%%%%%%%%%%%%%%%%%%%%%%%%%%%%%%%%%%%
%% PREAMBLE.
%%%%%%%%%%%%%%%%%%%%%%%%%%%%%%%%%%%%%%%%%%%%%%%%%%%%%%%%%%%%%%%%%

% Long table for page layout.
\usepackage{longtable}

% For side-by-side columns
\usepackage{multicol}

\newcommand{\parbreak}{\vspace{\baselineskip}}
\addtolength{\parskip}{\baselineskip}

% Geometry package for page margins.
\usepackage[
top=0.7in,
left=0.7in,
right=0.7in,
bottom=0.65in,
includefoot]
{geometry}

% XeLaTeX packages.
\usepackage{fontspec}
\defaultfontfeatures{Ligatures=TeX}
\usepackage{xunicode}
\usepackage{xltxtra}

% Font: Use "Tinos" as the main typeface (\textnormal{}, \normalfont).
% The "Tinos" fonts are released under the Apache License Version 2.0,
% and can be downloaded for free at <http://www.fontsquirrel.com/fonts/tinos>.
% Symbol table: <http://www.fileformat.info/info/unicode/font/tinos/grid.htm>
\setmainfont
[Path=./fonts/Tinos/,
ItalicFont=Tinos-Italic,
BoldFont=Tinos-Bold,
BoldItalicFont=Tinos-BoldItalic]
{Tinos-Regular.ttf}

% Sans-serif font: Switched to "Tinos".
\renewcommand{\sffamily}{\rmfamily}

% Typewriter (monospace) font: Switched to "Tinos".
\renewcommand{\ttfamily}{\rmfamily}

% Small caps font: Switched to "Tinos".
\renewcommand{\scshape}{\rmfamily}

% Secondary font: "GNU FreeFont".
% The "GNU FreeFont" fonts are released under the
% GNU General Public License Version 3, and can be downloaded
% for free at <https://savannah.gnu.org/projects/freefont/>.
\newcommand{\UseSecondaryFont}{\fontspec
[Path=./Fonts/GNUFreeFont/,
ItalicFont=FreeSerifItalic,
BoldFont=FreeSerifBold,
BoldItalicFont=FreeSerifBoldItalic]
{FreeSerif.otf}}

% Symbols (unicode).
\newcommand{\BulletSymbol}{\char"25CF}

% PDF settings and properties.
\usepackage{hyperref}

% Headers and footers: Blank header, page number in footer.
\makeatletter
\def\ps@plain{%
\def\@oddhead{}%
\def\@evenhead{}%
\def\@oddfoot{\footnotesize\hfill{\thepage}\hfill}%
\def\@evenfoot{\footnotesize\hfill{\thepage}\hfill}}
\makeatother

\pagestyle{plain}

% Paragraph style: No indentation.
\setlength{\parindent}{0in}

% Footnotes: Use symbols instead of numbers for labels.
\renewcommand{\thefootnote}{\fnsymbol{footnote}}

% Abbreviations for months.
\newcommand{\LongMonth}[1]{%
\ifcase#1\relax
\or January%
\or February%
\or March%
\or April%
\or May%
\or June%
\or July%
\or August%
\or September%
\or October%
\or November%
\or December%
\fi}

\newcommand{\Datestamp}[3]{\mbox{\LongMonth{#2} #1}}

% Macro: title (name).
\renewcommand{\title}[1]{%
\pdfbookmark{#1}{#1}%
\par\begin{center}%
\par\begin{Huge}%
\textbf{#1}%
\par\end{Huge}%
\par\end{center}%
\par\vspace{-1.75em}\par}

% Macro: subtitle (personal information below name).
\newenvironment{subtitle}
{
\par\begin{center}%
\par\begin{small}
}
{
\par\end{small}%
\par\end{center}
\par
}

% Macro: body (rest of the document).
\newenvironment{body}
{\par\vspace{-1em}\par
\begin{longtable}{p{0.15\textwidth}p{0.80\textwidth}}}
{\par\end{longtable}\par}

% Macro: section (new section for Education, Research Experience, etc.).
\renewcommand{\section}[2]{\\[-1em]\pdfbookmark{#1}{#1}\\%
\fontsize{9pt}{11pt}\selectfont\raggedright\textbf{\MakeUppercase{#2}}&}

% Macro: detail (text in smaller font under an entry).
\newenvironment{detail}{\small\parbreak}

% PDF settings and properties.
\hypersetup{
pdftitle={CV},
pdfauthor={Peter Goldsborough},
pdfsubject={http://www.thecodeinn.blogspot.com},
pdfcreator={},
pdfproducer={},
pdfkeywords={},
pdfpagemode={},
bookmarks=true,
unicode=true,
bookmarksopen=true,
pdfstartview=FitH,
pdfpagelayout=OneColumn,
pdfpagemode=UseOutlines,
hidelinks,
breaklinks}
